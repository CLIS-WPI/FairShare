%%%%%%%%%%%%%%%%%%%%%%%%%%%%%%
% ============================
% SECTION 4 — SYSTEM MODEL
% ============================
\section{System Model}

We consider a downlink multi-operator LEO NTN where $O$ satellite operators
share a common spectrum pool of total bandwidth $W$ (Hz). The system operates
in discrete time slots $t \in \mathcal{T}$, where channel states, visibility,
and beam associations evolve with satellite motion.

% -----------------------------
\subsection{Network Topology}

Let $\mathcal{S}=\{1,\dots,S\}$ denote the set of visible LEO satellites and
$\mathcal{U}=\{1,\dots,U\}$ the set of user terminals (UTs) on the ground.
Each satellite $s$ forms a multibeam footprint with beam set
$\mathcal{B}_s=\{1,\dots,B_s\}$, and we define the global beam set
\[
\mathcal{B} = \bigcup_{s\in\mathcal{S}} \mathcal{B}_s.
\]

Each user $u \in \mathcal{U}$ is served by beam $b(u,t)\in\mathcal{B}$ at time
$t$, with association changes induced by orbital motion. Users are partitioned
across operators through a mapping $o(u)\in\{1,\dots,O\}$.

% -----------------------------
\subsection{Shared Spectrum and Resource Allocation}

At each time slot $t$, operator $o$ receives bandwidth $w_o(t)$ satisfying
\begin{equation}
\sum_{o=1}^{O} w_o(t) \le W,\qquad w_o(t)\ge 0.
\end{equation}

Within each operator, per-user bandwidth $b_u(t)$ is allocated to active users
$\mathcal{U}_o(t)$:
\begin{equation}
\sum_{u\in\mathcal{U}_o(t)} b_u(t) \le w_o(t).
\end{equation}

Each beam $(s,b)$ has a power limit $P^{\max}_{s,b}$:
\begin{equation}
\sum_{u\in\mathcal{U}_{s,b}(t)} p_{s,b,u}(t) \le P^{\max}_{s,b}.
\end{equation}

These allocations are determined by the DSS policy under evaluation
(e.g., static slicing, priority-based sharing, load-based sharing).

% -----------------------------
\subsection{Channel and Propagation Model}

For user $u$ served by beam $(s,b)$, the complex channel coefficient is
\begin{equation}
h_u(t)
= g_u(t)\, G_{s,b}(t)\, L^{-1/2}(d_u(t))\,
e^{j 2\pi f_{D,u}(t)t},
\end{equation}
where $g_u(t)$ is small-scale fading, $G_{s,b}(t)$ is the antenna gain,
$L(d_u(t))$ is large-scale path-loss, and $f_{D,u}(t)$ is LEO Doppler shift
(up to tens of kHz in Ka-band).

The received signal is:
\begin{equation}
y_u(t)
= h_u(t)x_u(t)
+ \sum_{(s',b')\neq(s,b)} h_u^{(s',b')}(t)x_{u}^{(s',b')}(t)
+ n_u(t),
\end{equation}
where $n_u(t)$ is AWGN.

The corresponding SINR is:
\begin{equation}
\gamma_u(t)
= \frac{p_{s,b,u}(t)|h_u(t)|^2}
{\sum_{(s',b')\neq(s,b)} I_u^{(s',b')}(t) + N_0 b_u(t)}.
\end{equation}

% -----------------------------
\subsection{QoS Indicators}

Given allocated bandwidth $b_u(t)$, the achievable rate is
\begin{equation}
R_u(t) = b_u(t)\log_2\!\left(1+\gamma_u(t)\right).
\end{equation}

Latency $D_u(t)$ is computed via queueing delay or flow-level models.
Outage is defined as
\begin{equation}
\chi_u(t) = \mathbf{1}\{\gamma_u(t) < \gamma^{\text{th}}_u\}.
\end{equation}

Each user has a priority label $\pi_u\in\Pi$ (e.g., emergency, normal, best-effort).

% -----------------------------
\subsection{LEO Context Variables}

Consistent with prior NTN literature, we extract three context indicators:

\begin{itemize}
    \item \textbf{Doppler magnitude:} $\left|f_{D,u}(t)\right|$  
    \item \textbf{Elevation angle:} $\theta_u(t)$  
    \item \textbf{Beam load:}
    \begin{equation}
    \ell_{s,b}(t)
    = \frac{\sum_{u\in\mathcal{U}_{s,b}(t)} R_u(t)}
    {C_{s,b}(t)},
    \end{equation}
    where $C_{s,b}(t)$ is beam capacity.
\end{itemize}

Together, each user’s context vector is:
\[
\mathbf{k}_u(t)
=\left(
R_u(t),\, D_u(t),\, \mathbb{P}\{\chi_u(t)=1\},\,
\pi_u,\, |f_{D,u}(t)|,\, \theta_u(t),\, \ell_{s,b}(t)
\right).
\]



% ============================
% SECTION 5 — PROBLEM FORMULATION
% ============================
\section{Problem Formulation}

DSS policies determine the bandwidth and power allocations
$\{b_u(t), p_{s,b,u}(t)\}$, which in turn influence user QoS and fairness.
Our aim is to evaluate the fairness of an arbitrary DSS policy under
realistic LEO dynamics.

% -----------------------------
\subsection{Limitations of Classical Fairness Metrics}

Classical fairness metrics—Jain’s index, $\alpha$-fairness, max–min fairness—
are based solely on user rates:
\[
J(t)=\frac{\left(\sum_{u} R_u(t)\right)^2}{U\sum_u R_u^2(t)}.
\]

However, LEO dynamics introduce context-dependent inequalities:

\begin{itemize}
    \item users at low elevation suffer stronger path-loss;
    \item high Doppler induces unstable SINR and temporal unfairness;
    \item beam load varies rapidly with satellite footprint motion;
    \item multi-operator competition yields asymmetric QoS.
\end{itemize}

Thus, two allocations may exhibit identical $\{R_u(t)\}$ but differ widely in
fairness once context is considered—a failure repeatedly observed in recent
NTN DSS studies~\cite{ahmad2024leoscheduling,ahmed2025overview}.

% -----------------------------
\subsection{Context-Aware Fairness Objective}

We therefore define fairness as a function of both QoS and LEO context:

\begin{equation}
\mathcal{F}(t)
= F\Big( R_u(t), D_u(t), p^{\mathrm{out}}_u(t),
\pi_u, |f_{D,u}(t)|, \theta_u(t), \ell_{s,b}(t)\Big).
\label{eq:fuzzy-objective}
\end{equation}

The goal is:

\[
\textbf{Evaluate fairness of a DSS policy using a multi-criteria,
LEO-aware framework that remains valid under non-stationary channels.}
\]

Since $F(\cdot)$ is highly nonlinear and involves heterogeneous variables
(including discrete priorities and continuous Doppler), classical analytical
forms are unsuitable. This motivates a fuzzy-logic-based evaluator.

% -----------------------------
\subsection{Fuzzy Fairness Evaluation Problem}

For each user, we compute a fuzzy fairness score
\begin{equation}
\varphi_u(t)=F_{\mathrm{FIS}}\!\left(\mathbf{k}_u(t)\right),
\end{equation}
where $F_{\mathrm{FIS}}$ is a fuzzy inference system with:

\begin{itemize}
    \item membership functions for each QoS/context variable,
    \item a rule base encoding fairness semantics,
    \item centroid-based defuzzification.
\end{itemize}

The network-level fairness score is:
\begin{equation}
F_{\mathrm{fuzzy}}(t)
= \frac{\sum_{u=1}^{U} w_u\, \varphi_u(t)}
       {\sum_{u=1}^{U} w_u},
\end{equation}
where $w_u$ is a priority-derived weight.

This formulation allows policy-independent, interpretable fairness evaluation
under realistic LEO NTN conditions.


%%%%%%%%%%%%%%%%%%%%%%%%%%%%%
% ============================
% SECTION — FAIRNESS METRICS
% ============================
\section{Fairness Metrics}
\label{sec:fairness}

We evaluate fairness in dynamic spectrum sharing (DSS) using two complementary perspectives:
(i) classical allocation-based fairness indices, and
(ii) a new context-aware fuzzy fairness metric that captures LEO-specific inequalities.
This section formalizes both families of metrics.

% ============================
\subsection{Per-User QoS Indicators}
Each user $u$ observes the following QoS values at time $t$:
\[
R_u(t),\quad D_u(t),\quad p^{\mathrm{out}}_u(t)=\mathbb{P}\{\chi_u(t)=1\}.
\]
Users also belong to a service priority class $\pi_u \in \Pi$
mapped to a non-negative weight $w_u$.

To unify multi-dimensional QoS, each indicator is normalized to $[0,1]$ using
monotonic utility functions.

\paragraph{Rate utility}
\begin{equation}
u^{(R)}_u(t) = \min\!\left(1,\frac{R_u(t)}{R^{\mathrm{tar}}_u}\right),
\end{equation}
where $R^{\mathrm{tar}}_u$ is service target (possibly context-dependent).

\paragraph{Latency utility}
\begin{equation}
u^{(D)}_u(t)=
\begin{cases}
1, & D_u(t)\le D^{\mathrm{tar}}_u,\\
1-\dfrac{D_u(t)-D^{\mathrm{tar}}_u}{D^{\max}_u-D^{\mathrm{tar}}_u}, & \text{otherwise},\\
\end{cases}
\end{equation}

\paragraph{Outage utility}
\begin{equation}
u^{(O)}_u(t)=\max\!\left(0,1-\frac{p^{\mathrm{out}}_u(t)}{p^{\mathrm{tar}}_u}\right).
\end{equation}

These utilities are directly computable in simulation.

% ============================
% ============================
% REVISED SUBSECTION IN SYSTEM MODEL
% ============================

% ============================
% REVISED SUBSECTION IN SYSTEM MODEL
% ============================

\subsection{Context-Aware Channel Model}
The variability of LEO links is dominated by orbital dynamics, which induce rapid changes in slant-range distance, elevation angle, path loss, and Doppler shift. For a user $u$ associated with satellite $s$ at time $t$, the physical channel gain is explicitly expressed as
\begin{equation}
|h_u(t)|^2 =
G_{\mathrm{tx}}\!\left(\theta_u(t)\right)
G_{\mathrm{rx}}\!\left(\theta_u(t)\right)
\left(\frac{c}{4\pi f_c d_u(t)}\right)^2
\chi_{\mathrm{atm}}(t),
\label{eq:channel_phys}
\end{equation}
where $\theta_u(t)$ is the elevation angle, $d_u(t)$ is the slant-range distance, 
$f_c$ is the carrier frequency, and $\chi_{\mathrm{atm}}(t)$ captures atmospheric attenuation (e.g., rain fade).  
Users at low elevation simultaneously experience (i) reduced antenna gain and  
(ii) increased free-space path loss due to larger $d_u(t)$—a fundamental physical inequality that cannot be mitigated solely through scheduling or power control.


% ============================
% REVISED SUBSECTION IN FAIRNESS METRICS
% ============================

\subsection{Context-Aware Normalized Service Level}
\label{subsec:context_norm}

Classical fairness metrics implicitly assume identical channel potential among all users. 
In LEO systems, however, a user at low elevation or under adverse atmospheric conditions cannot physically achieve the same spectral efficiency as a user at the sub-satellite point, even under identical resource allocation.  
To isolate \emph{physical limitations} from \emph{resource-allocation fairness}, 
we define the \textbf{Context-Aware Reference Capacity} $C^{\mathrm{ref}}_u(t)$.

For each user $u$, we consider a hypothetical reference resource assignment consisting of a reference bandwidth $b^{\mathrm{ref}}$ and reference transmit power $p^{\mathrm{ref}}$.  
The corresponding single-user SINR is
\begin{equation}
\gamma^{\mathrm{ref}}_u(t)
=
\frac{p^{\mathrm{ref}} |h_u(t)|^2}
{N_0 b^{\mathrm{ref}} + I_{\mathrm{typ}}},
\label{eq:gamma_ref}
\end{equation}
where $I_{\mathrm{typ}}$ denotes a typical interference floor expected under multi-operator coexistence.

The \emph{maximum physically feasible capacity} user $u$ could obtain at time $t$ is then
\begin{equation}
C^{\mathrm{ref}}_u(t)
=
b^{\mathrm{ref}}
\log_2\!\left(1 + \gamma^{\mathrm{ref}}_u(t)\right).
\label{eq:c_ref}
\end{equation}

We define the \textbf{Context-Aware Normalized Service Level} as
\begin{equation}
\eta_u(t)
=
\min\!\left(
1,\;
\frac{R_u(t)}{\kappa\, C^{\mathrm{ref}}_u(t)}
\right),
\label{eq:eta_metric}
\end{equation}
where $\kappa$ scales the reference capacity to the actual system bandwidth.

\textit{Interpretation:}  
$\eta_u(t)$ quantifies how well the DSS policy satisfies user $u$ \emph{relative to their instantaneous channel geometry}.  
If a disadvantaged user (low elevation, long slant range, or atmospheric fade) achieves $\eta_u \approx 1$, then the allocation is fair—even if their absolute throughput is low—whereas classical fairness indices would erroneously classify this as unfair.

% ============================
\subsection{Classical Baselines: Rate-Only Fairness}
Classical fairness indices operate only on $\eta_u(t)$ and ignore LEO context.

\paragraph{Jain’s index}
\begin{equation}
J(t)=\frac{\left(\sum_{u=1}^{U}\eta_u(t)\right)^2}
{U\sum_{u=1}^{U}\eta_u^2(t)}.
\end{equation}

\paragraph{Weighted $\alpha$-fair utility}
\begin{equation}
U_\alpha(t)=\sum_{u=1}^{U} w_u\,
\phi_\alpha(\eta_u(t)),
\end{equation}
where
\[
\phi_\alpha(x)=
\begin{cases}
\log(x+\epsilon), & \alpha=1,\\
\dfrac{(x+\epsilon)^{1-\alpha}-1}{1-\alpha}, & \alpha\ne 1.
\end{cases}
\]

\textbf{Limitation.}
These indices treat two users with identical $\eta_u(t)$ as equally treated
even if one operates at low elevation with $40$ kHz Doppler and the other at
zenith with ideal geometry—an issue reported repeatedly in NTN literature.

% ============================
\subsection{Multi-Dimensional LEO Context Vector}

To incorporate physical-layer context, each user is represented by:
\begin{equation}
\mathbf{z}_u(t)=(
\eta_u(t),
u^{(D)}_u(t),
u^{(O)}_u(t),
\tilde{f}_{D,u}(t),
\tilde{\theta}_u(t),
\tilde{\ell}_{s,b}(t),
w_u
).
\label{eq:zvector}
\end{equation}

The normalization terms are:
\[
\tilde{f}_{D,u}(t)=\min\!\left(1,\frac{|f_{D,u}(t)|}{f_D^{\max}}\right),
\quad
\tilde{\theta}_u(t)=\frac{\theta_u(t)-\theta^{\min}}{\theta^{\max}-\theta^{\min}},
\]
\[
\tilde{\ell}_{s,b}(t)=
\min\!\left(1,\frac{\ell_{s,b}(t)}{\ell^{\max}}\right).
\]

% ============================
% ============================
% SECTION — MEMBERSHIP FUNCTIONS
% ============================
\section{Membership Function Design}

Each of the seven fuzzy inputs is mapped into three linguistic labels
(Low/Medium/High or equivalent). To enable reproducible implementation,
we specify the triangular membership functions (TMFs) used in all experiments.

\subsection{Normalized Input Range}
All inputs are normalized into $[0,1]$ before fuzzification:
\[
x \in [0,1].
\]
A generic triangular MF is defined as
\[
\mu_{\triangle}(x;a,b,c)
=
\begin{cases}
0, & x \le a, \\
\dfrac{x-a}{b-a}, & a < x \le b,\\[2mm]
\dfrac{c-x}{c-b}, & b < x < c,\\
0, & x \ge c.
\end{cases}
\]

\subsection{Fuzzy Sets}

\paragraph{Throughput ($T$)}
\[
\text{Low}: (0,0,0.4),\quad
\text{Med}: (0.2,0.5,0.8),\quad
\text{High}: (0.6,1,1)
\]

\paragraph{Latency ($D$)}
Low delay = good performance.
\[
\text{Good}: (0,0,0.3),\quad
\text{Acc}: (0.2,0.5,0.8),\quad
\text{Poor}: (0.6,1,1)
\]

\paragraph{Outage ($O$)}
\[
\text{Rare}: (0,0,0.2),\quad
\text{Occ}: (0.1,0.4,0.7),\quad
\text{Freq}: (0.5,1,1)
\]

\paragraph{Priority ($P$)}
\[
\text{Low}: (0,0,0.3),\;
\text{Norm}:(0.2,0.5,0.8),\;
\text{High}:(0.6,1,1)
\]

\paragraph{Doppler ($F_D$)}
(after normalization using $f_D^{\max}$)
\[
\text{Low}: (0,0,0.3),\quad
\text{Med}: (0.2,0.5,0.7),\quad
\text{High}: (0.5,1,1)
\]

\paragraph{Elevation ($\Theta$)}
\[
\text{Low}: (0,0,0.3),\quad
\text{Med}: (0.2,0.5,0.8),\quad
\text{High}: (0.6,1,1)
\]

\paragraph{Beam Load ($L$)}
\[
\text{Light}: (0,0,0.3),\quad
\text{Mod}: (0.2,0.5,0.8),\quad
\text{Heavy}: (0.6,1,1)
\]

\paragraph{Output Fairness}
\[
\text{Very-Low}:(0,0,0.2),\;
\text{Low}:(0.1,0.3,0.45),\;
\text{Med}:(0.3,0.5,0.7),\;
\text{High}:(0.6,0.8,1)
\]

% ============================
% SECTION — RULE BASE
% ============================
\section{Rule Base}

We construct a compact yet expressive rule base of 15 rules (compressing
hundreds of possible combinations). These rules encode NTN fairness semantics:

\begin{itemize}[leftmargin=3mm]
\item[\textbf{R1}] IF D is Poor AND O is Freq THEN Fairness is Very-Low.
\item[\textbf{R2}] IF T is Low AND L is Heavy THEN Fairness is Low.
\item[\textbf{R3}] IF F$_D$ is High AND $\Theta$ is Low THEN Fairness is Low.
\item[\textbf{R4}] IF P is High AND O is Rare THEN Fairness is High.
\item[\textbf{R5}] IF P is Low AND O is Freq THEN Fairness is Very-Low.
\item[\textbf{R6}] IF T is Med AND L is Light THEN Fairness is Med.
\item[\textbf{R7}] IF T is High AND D is Good THEN Fairness is High.
\item[\textbf{R8}] IF $\Theta$ is High AND F$_D$ is Low THEN Fairness is High.
\item[\textbf{R9}] IF $\Theta$ is Low AND D is Poor THEN Fairness is Low.
\item[\textbf{R10}] IF L is Heavy AND F$_D$ is High THEN Fairness is Low.
\item[\textbf{R11}] IF P is High AND T is Med THEN Fairness is High.
\item[\textbf{R12}] IF P is Norm AND T is Low THEN Fairness is Low.
\item[\textbf{R13}] IF O is Rare AND D is Good THEN Fairness is High.
\item[\textbf{R14}] IF T is High AND L is Heavy THEN Fairness is Med.
\item[\textbf{R15}] IF all conditions are Medium THEN Fairness is Med.
\end{itemize}

This rule base was optimized to maximize distinguishability between classical
fairness and fuzzy fairness under LEO context imbalance.

% ============================
% ALGORITHM BLOCK
% ============================
\section{Algorithm: Fuzzy Fairness Computation}

\begin{algorithm}[h!]
\caption{Fuzzy Fairness Evaluation for DSS in LEO Networks}
\begin{algorithmic}[1]
\State \textbf{Input:} QoS $\{R_u,D_u,p^{out}_u\}$, Context $\{|f_{D,u}|,\theta_u,\ell_{s,b}\}$, Priority $w_u$
\For{each user $u$}
    \State Normalize inputs $\rightarrow x_u \in[0,1]^7$
    \State Compute MF degrees for all fuzzy sets
    \State Evaluate rule base $\mathcal{R}$ to obtain $\mu_r(z)$
    \State Aggregate: $\mu_{\text{agg}}(z)=\max_r \mu_r(z)$
    \State Defuzzify via centroid:
    \[
    \varphi_u=\frac{\int_0^1 z\,\mu_{\text{agg}}(z)dz}
    {\int_0^1 \mu_{\text{agg}}(z)dz}
    \]
\EndFor
\State \textbf{return} network fairness:
\[
F_{\mathrm{fuzzy}}=\frac{\sum_u w_u\varphi_u}{\sum_u w_u}
\]
\end{algorithmic}
\end{algorithm}

% ============================
% COMPLEXITY AND IMPLEMENTATION
% ============================
\section{Complexity and Implementation Notes}

For $U$ users, $M$ membership functions, and $R$ rules:
\[
\mathcal{O}(UM + UR)
\]
Since $M=21$ (7 inputs × 3 MFs) and $R=15$, the system runs easily in real time
($<1$ ms per decision), enabling on-board satellite or gateway deployment.

% ============================
% ADDITIONAL FIGURES
% ============================
\section{Additional Figures}

\begin{figure}[t]
\centering
\includegraphics[width=\columnwidth]{system_diagram_fuzzy_dss.pdf}
\caption{System overview of the fuzzy fairness evaluator within a DSS loop.}
\end{figure}

\begin{figure}[t]
\centering
\includegraphics[width=\columnwidth]{fuzzy_surface_plot.pdf}
\caption{Example fuzzy surface: fairness as a function of elevation and Doppler.}
\end{figure}

% ============================
% DISCUSSION AND LIMITATIONS
% ============================
\section{Discussion and Limitations}

The fuzzy fairness score reveals disparities hidden by classical indices.  
However:

\begin{itemize}
\item Fuzzy rules require expert-defined semantics (not fully automated).
\item Model accuracy depends on input normalization (context ranges must be realistic).
\item Beam-level interference modeling is simplified.
\end{itemize}

These limitations can be addressed by integrating reinforcement learning to
auto-tune rule weights and normalization parameters.

% ============================
% CONCLUSION
% ============================
\section{Conclusion}
This work presented a context-aware fuzzy fairness evaluation framework for
multi-operator LEO DSS. The method integrates QoS indicators and physical-layer
context, enabling interpretable and reliable fairness assessment under LEO
dynamics. Experiments show that classical fairness indices can overestimate
fairness by up to 28\%, while the proposed metric correctly captures
Doppler-, elevation-, and load-induced disparities. This framework provides a
scalable, explainable tool for multi-operator coexistence validation in future
NTN/6G systems.



% ============================
% ACK
% ============================
\section*{Acknowledgment}
\noindent
This work was supported by the NTIA (Award No. 51-60-IF007) and NSF (Award No. CNS-2120442).

% ============================
% REFERENCES
% ============================
\enlargethispage{\baselineskip}
\bibliographystyle{IEEEtran}
\bibliography{./bib/main.bib}

\end{document}